\documentclass{beamer}
\usetheme{Boadilla}
\usefonttheme{serif}

\AtBeginEnvironment{frame}{\setcounter{footnote}{0}}
\addtobeamertemplate{footnote}{}{\vspace{2ex}}

\usepackage{amsmath,amsthm,amssymb}
\usepackage{mathtext}

\usepackage[T2A]{fontenc}
\usepackage[utf8]{inputenc}
\usepackage[russian]{babel}

\usepackage{csquotes}

%\graphicspath{ {./materials/} }

\title[competition01]{Решение задачи соревнования \\ по курсу \enquote{Машинное обучение}}
\subtitle{}
\author{Денисов Д.М.}
\institute[мехмат МГУ]{Механико-математический факультет МГУ имени М.В. Ломоносова}
\date{}

\begin{document}
	\begin{frame}
		\titlepage
	\end{frame}

	\begin{frame}
		\frametitle{Восстановление пропусков}
		Данные разнородны, заменять на одно значение смысла мало \newline
		$\implies$ используем KNN \newline
		
		Параметры: 5 соседей, веса пропорциональны расстояниям.
	\end{frame}

	\begin{frame}
		\frametitle{Настройка классификатора}
		Выбираем классификатор KNN и проводим кросс-валидацию на случайной десятой части тренировочной выборки по следующей сетке гиперпараметров:
		\begin{alignat*}{3}
			&algorithm &&: [kd\_tree, ball\_tree] \\
			&n\_neighbors &&: [1, 3, 5, 7, 9] \\
			&weights &&: [distance]
			\intertext{Получаем:}
			&algorithm &&: kd\_tree \\
			&n\_neighbors &&: 9 \\
			&weights &&: distance
		\end{alignat*}
		
		Количество соседей получилось равным максимальному \newline
		$\implies$ может быть и больше...
	\end{frame}

	\begin{frame}
		\frametitle{Настройка классификатора}
		Меняем сетку гиперпараметров, увеличивая максимальное число соседей:
		\begin{alignat*}{3}
			&algorithm &&: [kd\_tree, ball\_tree] \\
			&n\_neighbors &&: [1, \ldots, 20] \\
			&weights &&: [distance]
			\intertext{Получаем:}
			&algorithm &&: kd\_tree \\
			&n\_neighbors &&: 13 \\
			&weights &&: distance
		\end{alignat*}
		
		Действительно, 13 соседей подходят больше.
	\end{frame}

	\begin{frame}
		\frametitle{Обучение}
		Обучаем на десятой части тренировочной выборки \newline
		$\implies$ accuracy $\sim\! 90,5\%$. \newline
		
		Обучаем на всей тренировочной выборке \newline
		$\implies$ accuracy $\sim\! 95,4\%$.
	\end{frame}
\end{document}